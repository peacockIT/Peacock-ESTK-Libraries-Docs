%% Generated by Sphinx.
\def\sphinxdocclass{report}
\documentclass[a4paper,11pt,openany,oneside,english]{sphinxmanual}
\ifdefined\pdfpxdimen
   \let\sphinxpxdimen\pdfpxdimen\else\newdimen\sphinxpxdimen
\fi \sphinxpxdimen=.75bp\relax

\PassOptionsToPackage{warn}{textcomp}
\usepackage[utf8]{inputenc}
\ifdefined\DeclareUnicodeCharacter
% support both utf8 and utf8x syntaxes
  \ifdefined\DeclareUnicodeCharacterAsOptional
    \def\sphinxDUC#1{\DeclareUnicodeCharacter{"#1}}
  \else
    \let\sphinxDUC\DeclareUnicodeCharacter
  \fi
  \sphinxDUC{00A0}{\nobreakspace}
  \sphinxDUC{2500}{\sphinxunichar{2500}}
  \sphinxDUC{2502}{\sphinxunichar{2502}}
  \sphinxDUC{2514}{\sphinxunichar{2514}}
  \sphinxDUC{251C}{\sphinxunichar{251C}}
  \sphinxDUC{2572}{\textbackslash}
\fi
\usepackage{cmap}
\usepackage[T1]{fontenc}
\usepackage{amsmath,amssymb,amstext}
\usepackage[ngerman]{babel}



\usepackage{times}
\expandafter\ifx\csname T@LGR\endcsname\relax
\else
% LGR was declared as font encoding
  \substitutefont{LGR}{\rmdefault}{cmr}
  \substitutefont{LGR}{\sfdefault}{cmss}
  \substitutefont{LGR}{\ttdefault}{cmtt}
\fi
\expandafter\ifx\csname T@X2\endcsname\relax
  \expandafter\ifx\csname T@T2A\endcsname\relax
  \else
  % T2A was declared as font encoding
    \substitutefont{T2A}{\rmdefault}{cmr}
    \substitutefont{T2A}{\sfdefault}{cmss}
    \substitutefont{T2A}{\ttdefault}{cmtt}
  \fi
\else
% X2 was declared as font encoding
  \substitutefont{X2}{\rmdefault}{cmr}
  \substitutefont{X2}{\sfdefault}{cmss}
  \substitutefont{X2}{\ttdefault}{cmtt}
\fi


\usepackage[Bjarne]{fncychap}
\usepackage{sphinx}

\fvset{fontsize=\small}
\usepackage{geometry}

% Include hyperref last.
\usepackage{hyperref}
% Fix anchor placement for figures with captions.
\usepackage{hypcap}% it must be loaded after hyperref.
% Set up styles of URL: it should be placed after hyperref.
\urlstyle{same}
\addto\captionsenglish{\renewcommand{\contentsname}{Getting Started}}

\usepackage{sphinxmessages}
\setcounter{tocdepth}{1}

\setcounter{tocdepth}{2}

\title{Lorem Ipsum Dokumentation}
\date{Jun 03, 2019}
\release{0.0.1}
\author{Daniel Kreth}
\newcommand{\sphinxlogo}{\sphinxincludegraphics{AE_Preferences.png}\par}
\renewcommand{\releasename}{Release}
\makeindex
\begin{document}

\pagestyle{empty}
\sphinxmaketitle
\pagestyle{plain}
\sphinxtableofcontents
\pagestyle{normal}
\phantomsection\label{\detokenize{index::doc}}


A documentation for the Peacock ESTK Library.


\chapter{Quick References}
\label{\detokenize{index:quick-references}}

\section{About This Documentation}
\label{\detokenize{intro:about-this-documentation}}\label{\detokenize{intro::doc}}

\subsection{Description}
\label{\detokenize{intro:description}}
To be written.

If you’re starting out with the Peacock ESTK Library,
read the {\hyperref[\detokenize{getting_started/getting_started::doc}]{\sphinxcrossref{\DUrole{doc}{Getting Started}}}} section first.


\section{Installation}
\label{\detokenize{getting_started/install:installation}}\label{\detokenize{getting_started/install::doc}}
\begin{sphinxShadowBox}
\sphinxstyletopictitle{Table of Contents}
\begin{itemize}
\item {} 
\phantomsection\label{\detokenize{getting_started/install:id1}}{\hyperref[\detokenize{getting_started/install:installation}]{\sphinxcrossref{Installation}}}
\begin{itemize}
\item {} 
\phantomsection\label{\detokenize{getting_started/install:id2}}{\hyperref[\detokenize{getting_started/install:install-instructions}]{\sphinxcrossref{Install Instructions}}}

\item {} 
\phantomsection\label{\detokenize{getting_started/install:id3}}{\hyperref[\detokenize{getting_started/install:after-effects-script-folder}]{\sphinxcrossref{After Effects Script Folder}}}

\item {} 
\phantomsection\label{\detokenize{getting_started/install:id4}}{\hyperref[\detokenize{getting_started/install:appdata}]{\sphinxcrossref{appData}}}

\item {} 
\phantomsection\label{\detokenize{getting_started/install:id5}}{\hyperref[\detokenize{getting_started/install:userdata}]{\sphinxcrossref{userData}}}

\end{itemize}

\end{itemize}
\end{sphinxShadowBox}


\subsection{Install Instructions}
\label{\detokenize{getting_started/install:install-instructions}}

\subsection{After Effects Script Folder}
\label{\detokenize{getting_started/install:after-effects-script-folder}}
Copy the \sphinxstylestrong{Peacock ESTK Library} folder into one of the application folders as follows:


\subsection{appData}
\label{\detokenize{getting_started/install:appdata}}
In Windows, the value of \%APPDATA\% (by default, \sphinxcode{\sphinxupquote{C:\textbackslash{}Documents and Settings\textbackslash{}All Users\textbackslash{}Application Data}})
\begin{itemize}
\item {} 
(Windows) \sphinxcode{\sphinxupquote{C:\textbackslash{}Documents and Settings\textbackslash{}All Users\textbackslash{}Application Data\textbackslash{}Peacock ESTK Library}}

\item {} 
(Windows) \sphinxcode{\sphinxupquote{C:\textbackslash{}ProgramData\textbackslash{}Peacock ESTK Library}}

\item {} 
(Mac OS) \sphinxcode{\sphinxupquote{/Library/Application Support/Peacock ESTK Library}}

\end{itemize}


\subsection{userData}
\label{\detokenize{getting_started/install:userdata}}\begin{itemize}
\item {} 
(Windows) \sphinxcode{\sphinxupquote{C:\textbackslash{}Documents and Settings\textbackslash{}username\textbackslash{}Application Data\textbackslash{}Peacock ESTK Library}}

\item {} 
(Mac OS) \sphinxcode{\sphinxupquote{\textasciitilde{}/Library/Application Support/Peacock ESTK Library}}

\end{itemize}


\section{Getting Started}
\label{\detokenize{getting_started/getting_started:getting-started}}\label{\detokenize{getting_started/getting_started::doc}}
\begin{sphinxShadowBox}
\sphinxstyletopictitle{Table of Contents}
\begin{itemize}
\item {} 
\phantomsection\label{\detokenize{getting_started/getting_started:id1}}{\hyperref[\detokenize{getting_started/getting_started:getting-started}]{\sphinxcrossref{Getting Started}}}
\begin{itemize}
\item {} 
\phantomsection\label{\detokenize{getting_started/getting_started:id2}}{\hyperref[\detokenize{getting_started/getting_started:quick-start}]{\sphinxcrossref{Quick Start}}}

\item {} 
\phantomsection\label{\detokenize{getting_started/getting_started:id3}}{\hyperref[\detokenize{getting_started/getting_started:section-1}]{\sphinxcrossref{Section 1}}}
\begin{itemize}
\item {} 
\phantomsection\label{\detokenize{getting_started/getting_started:id4}}{\hyperref[\detokenize{getting_started/getting_started:section-2}]{\sphinxcrossref{Section 2}}}

\end{itemize}

\end{itemize}

\end{itemize}
\end{sphinxShadowBox}


\subsection{Quick Start}
\label{\detokenize{getting_started/getting_started:quick-start}}
Before you can start using the BpmSlicer script you need to set the checkbox of the following setting to true.

\sphinxcode{\sphinxupquote{After Effects -\textgreater{} Preferences -\textgreater{} General -\textgreater{} Allow Scripts To Write Files And Access Network}}
(\sphinxcode{\sphinxupquote{Skripte können Dateien schreiben und haben Netwerkzugang}})

\begin{figure}[htbp]
\centering

\noindent\sphinxincludegraphics{{AE_Preferences}.png}
\end{figure}


\subsection{Section 1}
\label{\detokenize{getting_started/getting_started:section-1}}

\subsubsection{Section 2}
\label{\detokenize{getting_started/getting_started:section-2}}

\section{Peacock After Effects Utility}
\label{\detokenize{modules/ae_utils/readme:peacock-after-effects-utility}}\label{\detokenize{modules/ae_utils/readme::doc}}

\section{Peacock Illustrator Utility}
\label{\detokenize{modules/ai_utils/readme:peacock-illustrator-utility}}\label{\detokenize{modules/ai_utils/readme::doc}}

\section{Peacock Base64}
\label{\detokenize{modules/base64/readme:peacock-base64}}\label{\detokenize{modules/base64/readme::doc}}

\section{Peacock BeatManager}
\label{\detokenize{modules/beatmanager/readme:peacock-beatmanager}}\label{\detokenize{modules/beatmanager/readme::doc}}

\section{Peacock Binary Creator}
\label{\detokenize{modules/binarycreator/readme:peacock-binary-creator}}\label{\detokenize{modules/binarycreator/readme::doc}}

\section{Peacock Composition Template}
\label{\detokenize{modules/comp_template/readme:peacock-composition-template}}\label{\detokenize{modules/comp_template/readme::doc}}

\section{Peacock Console}
\label{\detokenize{modules/console/readme:peacock-console}}\label{\detokenize{modules/console/readme::doc}}
\begin{sphinxShadowBox}
\sphinxstyletopictitle{Table of Contents}
\begin{itemize}
\item {} 
\phantomsection\label{\detokenize{modules/console/readme:id1}}{\hyperref[\detokenize{modules/console/readme:peacock-console}]{\sphinxcrossref{Peacock Console}}}
\begin{itemize}
\item {} 
\phantomsection\label{\detokenize{modules/console/readme:id2}}{\hyperref[\detokenize{modules/console/readme:tabcompletion}]{\sphinxcrossref{Tabcompletion}}}

\item {} 
\phantomsection\label{\detokenize{modules/console/readme:id3}}{\hyperref[\detokenize{modules/console/readme:shortcuts}]{\sphinxcrossref{Shortcuts}}}

\item {} 
\phantomsection\label{\detokenize{modules/console/readme:id4}}{\hyperref[\detokenize{modules/console/readme:peacock-commands}]{\sphinxcrossref{Peacock Commands}}}

\end{itemize}

\end{itemize}
\end{sphinxShadowBox}

The console is like a command line. Three different types of input are
possible.
\begin{description}
\item[{After Effects Keyframes / Mocha Tracking Data}] \leavevmode
You can either paste Mocha tracking data directly from Mocha into the console or Keyframes from a selected layer property in After Effects. Note that only Position, Scale and Rotation keyframes are supported yet. If you press: \sphinxcode{\sphinxupquote{Cmd+Enter}} or the \sphinxcode{\sphinxupquote{'R' button}} the keyframes are getting parsed into an internal keyframe data structure.

\begin{sphinxadmonition}{note}{Note:}
There is no use for the parsed keyframes yet. I plan to manipulate tracking data keyframes synced to the beat.
\end{sphinxadmonition}

\item[{Peacock midi note data}] \leavevmode
The external standalone program “Midiconverter” converts a midi file (.mid) into ‘Peacock midi note data’. For this to work the midi notes in the midi file have to be in the range from C3 - C4 and you need to set the proper bpm value.
After the midi file is converted the ‘Peacock midi note data’ is automatically copied to the clipboard and a .txt file with the same ‘Peacock midi note data’ is created as a sibling of the midi file.
The ‘Peacock midi note data’ can be directly pasted into the BpmSlicer console. By pressing \sphinxcode{\sphinxupquote{Cmd+Enter}} or the \sphinxcode{\sphinxupquote{'R' button}} the slice data is getting parsed into the internal slices array which can then be used to slice layers in a composition.

\item[{Executable javascript}] \leavevmode
You can write any javascript code you like and execute it directly from the console. Some useful code snippets are accessible through tabcompletion and shortcuts

\end{description}


\subsection{Tabcompletion}
\label{\detokenize{modules/console/readme:tabcompletion}}
A list of all tab completion code snippets.
\begin{description}
\item[{\sphinxcode{\sphinxupquote{for}}}] \leavevmode
\begin{sphinxVerbatim}[commandchars=\\\{\}]
\PYG{n+nx}{aeHelper}\PYG{p}{.}\PYG{n+nx}{selectAllLayers}\PYG{p}{(}\PYG{n+nx}{comp}\PYG{p}{)}\PYG{p}{;}
\PYG{k}{for}\PYG{p}{(}\PYG{k+kd}{var} \PYG{n+nx}{i}\PYG{o}{=}\PYG{l+m+mi}{0}\PYG{p}{;} \PYG{n+nx}{i}\PYG{o}{\PYGZlt{}}\PYG{n+nx}{comp}\PYG{p}{.}\PYG{n+nx}{selectedLayers}\PYG{p}{.}\PYG{n+nx}{length}\PYG{p}{;} \PYG{n+nx}{i}\PYG{o}{++}\PYG{p}{)}\PYG{p}{\PYGZob{}}
  \PYG{k+kd}{var} \PYG{n+nx}{layer} \PYG{o}{=} \PYG{n+nx}{comp}\PYG{p}{.}\PYG{n+nx}{selectedLayers}\PYG{p}{[}\PYG{n+nx}{i}\PYG{p}{]}\PYG{p}{;}
  \PYG{n+nx}{log}\PYG{p}{.}\PYG{n+nx}{appendLog}\PYG{p}{(}\PYG{n+nx}{i} \PYG{o}{+} \PYG{l+s+s2}{\PYGZdq{} \PYGZdq{}} \PYG{o}{+} \PYG{n+nx}{layer}\PYG{p}{.}\PYG{n+nx}{name}\PYG{p}{)}\PYG{p}{;}
\PYG{p}{\PYGZcb{}}
\end{sphinxVerbatim}

\item[{\sphinxcode{\sphinxupquote{fors}}}] \leavevmode
\begin{sphinxVerbatim}[commandchars=\\\{\}]
\PYG{k}{for}\PYG{p}{(}\PYG{k+kd}{var} \PYG{n+nx}{i}\PYG{o}{=}\PYG{l+m+mi}{0}\PYG{p}{;} \PYG{n+nx}{i}\PYG{o}{\PYGZlt{}}\PYG{n+nx}{slices}\PYG{p}{.}\PYG{n+nx}{slices}\PYG{p}{.}\PYG{n+nx}{length}\PYG{p}{;} \PYG{n+nx}{i}\PYG{o}{++}\PYG{p}{)} \PYG{p}{\PYGZob{}}
  \PYG{k+kd}{var} \PYG{n+nx}{slice} \PYG{o}{=} \PYG{n+nx}{slices}\PYG{p}{.}\PYG{n+nx}{slices}\PYG{p}{[}\PYG{n+nx}{i}\PYG{p}{]}\PYG{p}{;}
  \PYG{n+nx}{log}\PYG{p}{.}\PYG{n+nx}{appendLog}\PYG{p}{(}\PYG{n+nx}{i} \PYG{o}{+} \PYG{l+s+s2}{\PYGZdq{} \PYGZdq{}} \PYG{o}{+} \PYG{n+nx}{slice}\PYG{p}{.}\PYG{n+nx}{getInPoint}\PYG{p}{(}\PYG{p}{)}\PYG{p}{)}\PYG{p}{;}
\PYG{p}{\PYGZcb{}}
\PYG{n+nx}{slices}\PYG{p}{.}\PYG{n+nx}{slices}\PYG{p}{.}\PYG{n+nx}{length}\PYG{p}{;}
\end{sphinxVerbatim}

\item[{\sphinxcode{\sphinxupquote{form}}}] \leavevmode
\begin{sphinxVerbatim}[commandchars=\\\{\}]
\PYG{k}{for}\PYG{p}{(}\PYG{k+kd}{var} \PYG{n+nx}{i}\PYG{o}{=}\PYG{l+m+mi}{0}\PYG{p}{;} \PYG{n+nx}{i}\PYG{o}{\PYGZlt{}}\PYG{n+nx}{markers}\PYG{p}{.}\PYG{n+nx}{markers}\PYG{p}{.}\PYG{n+nx}{length}\PYG{p}{;} \PYG{n+nx}{i}\PYG{o}{++}\PYG{p}{)} \PYG{p}{\PYGZob{}}
  \PYG{k+kd}{var} \PYG{n+nx}{marker} \PYG{o}{=} \PYG{n+nx}{markers}\PYG{p}{.}\PYG{n+nx}{markers}\PYG{p}{[}\PYG{n+nx}{i}\PYG{p}{]}\PYG{p}{;}
  \PYG{n+nx}{log}\PYG{p}{.}\PYG{n+nx}{appendLog}\PYG{p}{(}\PYG{n+nx}{i} \PYG{o}{+} \PYG{l+s+s2}{\PYGZdq{} \PYGZdq{}} \PYG{o}{+} \PYG{n+nx}{marker}\PYG{p}{.}\PYG{n+nx}{getTime}\PYG{p}{(}\PYG{p}{)}\PYG{p}{)}\PYG{p}{;}
\PYG{p}{\PYGZcb{}}
\PYG{n+nx}{markers}\PYG{p}{.}\PYG{n+nx}{markers}\PYG{p}{.}\PYG{n+nx}{length}\PYG{p}{;}
\end{sphinxVerbatim}

\item[{\sphinxcode{\sphinxupquote{if}}}] \leavevmode
\begin{sphinxVerbatim}[commandchars=\\\{\}]
\PYG{k}{if}\PYG{p}{(}\PYG{n+nx}{markers}\PYG{p}{.}\PYG{n+nx}{markers}\PYG{p}{.}\PYG{n+nx}{length} \PYG{o}{\PYGZgt{}} \PYG{l+m+mi}{10}\PYG{p}{)} \PYG{p}{\PYGZob{}}
  \PYG{n+nx}{log}\PYG{p}{.}\PYG{n+nx}{appendLog}\PYG{p}{(}\PYG{l+s+s2}{\PYGZdq{}More than 10 markers exist\PYGZdq{}}\PYG{p}{)}\PYG{p}{;}
\PYG{p}{\PYGZcb{}}
\PYG{p}{(}\PYG{n+nx}{markers}\PYG{p}{.}\PYG{n+nx}{markers}\PYG{p}{.}\PYG{n+nx}{length} \PYG{o}{\PYGZgt{}} \PYG{l+m+mi}{10}\PYG{p}{)}\PYG{p}{;}
\end{sphinxVerbatim}

\item[{\sphinxcode{\sphinxupquote{if else}}}] \leavevmode
\begin{sphinxVerbatim}[commandchars=\\\{\}]
\PYG{k}{if}\PYG{p}{(}\PYG{n+nx}{markers}\PYG{p}{.}\PYG{n+nx}{markers}\PYG{p}{.}\PYG{n+nx}{length} \PYG{o}{\PYGZgt{}} \PYG{l+m+mi}{10}\PYG{p}{)} \PYG{p}{\PYGZob{}}
  \PYG{n+nx}{log}\PYG{p}{.}\PYG{n+nx}{appendLog}\PYG{p}{(}\PYG{l+s+s2}{\PYGZdq{}More than 10 markers exist\PYGZdq{}}\PYG{p}{)}\PYG{p}{;}
\PYG{p}{\PYGZcb{}}\PYG{k}{else} \PYG{p}{\PYGZob{}}
  \PYG{n+nx}{log}\PYG{p}{.}\PYG{n+nx}{appendLog}\PYG{p}{(}\PYG{l+s+s2}{\PYGZdq{}Less than 10 (or equal) markers exist\PYGZdq{}}\PYG{p}{)}\PYG{p}{;}
\PYG{p}{\PYGZcb{}}
\PYG{p}{(}\PYG{n+nx}{markers}\PYG{p}{.}\PYG{n+nx}{markers}\PYG{p}{.}\PYG{n+nx}{length} \PYG{o}{\PYGZgt{}} \PYG{l+m+mi}{10}\PYG{p}{)}\PYG{p}{;}
\end{sphinxVerbatim}

\end{description}


\subsection{Shortcuts}
\label{\detokenize{modules/console/readme:shortcuts}}
A list of all tab shortcut code snippets.
\begin{description}
\item[{\sphinxcode{\sphinxupquote{select}}}] \leavevmode
\begin{sphinxVerbatim}[commandchars=\\\{\}]
\PYG{k+kd}{var} \PYG{n+nx}{counter} \PYG{o}{=} \PYG{l+m+mi}{0}\PYG{p}{;}
\PYG{k}{for}\PYG{p}{(}\PYG{k+kd}{var} \PYG{n+nx}{i}\PYG{o}{=}\PYG{l+m+mi}{0}\PYG{p}{;} \PYG{n+nx}{i}\PYG{o}{\PYGZlt{}}\PYG{n+nx}{comp}\PYG{p}{.}\PYG{n+nx}{selectedLayers}\PYG{p}{.}\PYG{n+nx}{length}\PYG{p}{;}\PYG{n+nx}{i}\PYG{o}{++}\PYG{p}{)}\PYG{p}{\PYGZob{}}
  \PYG{k+kd}{var} \PYG{n+nx}{layer} \PYG{o}{=} \PYG{n+nx}{comp}\PYG{p}{.}\PYG{n+nx}{selectedLayers}\PYG{p}{[}\PYG{n+nx}{i}\PYG{p}{]}\PYG{p}{;}
  \PYG{k}{if}\PYG{p}{(}\PYG{n+nx}{layer}\PYG{p}{.}\PYG{n+nx}{name} \PYG{o}{!=} \PYG{l+s+s2}{\PYGZdq{}\PYGZdq{}}\PYG{p}{)}\PYG{p}{\PYGZob{}}
    \PYG{n+nx}{layer}\PYG{p}{.}\PYG{n+nx}{selected} \PYG{o}{=} \PYG{k+kc}{true}\PYG{p}{;}
  \PYG{p}{\PYGZcb{}}
  \PYG{n+nx}{counter}\PYG{o}{++}\PYG{p}{;}
\PYG{p}{\PYGZcb{}}
\PYG{n+nx}{counter}\PYG{p}{;}
\end{sphinxVerbatim}

\item[{\sphinxcode{\sphinxupquote{bpm}}}] \leavevmode
\begin{sphinxVerbatim}[commandchars=\\\{\}]
\PYG{n+nx}{beatManager}\PYG{p}{.}\PYG{n+nx}{setBpm}\PYG{p}{(}\PYG{l+m+mi}{166}\PYG{p}{)}\PYG{p}{;}
\PYG{n+nx}{beatManager}\PYG{p}{.}\PYG{n+nx}{getBpm}\PYG{p}{(}\PYG{p}{)}\PYG{p}{;}
\end{sphinxVerbatim}

\item[{\sphinxcode{\sphinxupquote{beatRate}}}] \leavevmode
\begin{sphinxVerbatim}[commandchars=\\\{\}]
\PYG{n+nx}{beatManager}\PYG{p}{.}\PYG{n+nx}{calculateBeatRate}\PYG{p}{(}\PYG{l+m+mi}{120}\PYG{p}{,} \PYG{l+s+s2}{\PYGZdq{}1/4\PYGZdq{}}\PYG{p}{)}\PYG{p}{;}
\end{sphinxVerbatim}

\item[{\sphinxcode{\sphinxupquote{status}}}] \leavevmode
\begin{sphinxVerbatim}[commandchars=\\\{\}]
\PYG{n+nx}{markers}\PYG{p}{.}\PYG{n+nx}{markers}\PYG{p}{.}\PYG{n+nx}{length} \PYG{o}{+} \PYG{l+s+s2}{\PYGZdq{} markers; \PYGZdq{}} \PYG{o}{+} \PYG{n+nx}{slices}\PYG{p}{.}\PYG{n+nx}{slices}\PYG{p}{.}\PYG{n+nx}{length} \PYG{o}{+} \PYG{l+s+s2}{\PYGZdq{} slices\PYGZdq{}}\PYG{p}{;}
\end{sphinxVerbatim}

\item[{\sphinxcode{\sphinxupquote{rename}}}] \leavevmode
\begin{sphinxVerbatim}[commandchars=\\\{\}]
\PYG{k+kd}{var} \PYG{n+nx}{name} \PYG{o}{=} \PYG{l+s+s2}{\PYGZdq{}newName\PYGZdq{}}\PYG{p}{;}
\PYG{n+nx}{re} \PYG{o}{=} \PYG{l+s+sr}{/\PYGZca{}newName/}\PYG{p}{;}
\PYG{n+nx}{aeHelper}\PYG{p}{.}\PYG{n+nx}{selectAllLayers}\PYG{p}{(}\PYG{n+nx}{comp}\PYG{p}{)}\PYG{p}{;}
\PYG{k+kd}{var} \PYG{n+nx}{counter} \PYG{o}{=} \PYG{l+m+mi}{0}\PYG{p}{;}
\PYG{k}{for}\PYG{p}{(}\PYG{k+kd}{var} \PYG{n+nx}{i}\PYG{o}{=}\PYG{l+m+mi}{0}\PYG{p}{;} \PYG{n+nx}{i}\PYG{o}{\PYGZlt{}}\PYG{n+nx}{comp}\PYG{p}{.}\PYG{n+nx}{selectedLayers}\PYG{p}{.}\PYG{n+nx}{length}\PYG{p}{;} \PYG{n+nx}{i}\PYG{o}{++}\PYG{p}{)}\PYG{p}{\PYGZob{}}
  \PYG{k+kd}{var} \PYG{n+nx}{layer} \PYG{o}{=} \PYG{n+nx}{comp}\PYG{p}{.}\PYG{n+nx}{selectedLayers}\PYG{p}{[}\PYG{n+nx}{i}\PYG{p}{]}\PYG{p}{;}
  \PYG{k}{if}\PYG{p}{(}\PYG{n+nx}{re}\PYG{p}{.}\PYG{n+nx}{test}\PYG{p}{(}\PYG{n+nx}{layer}\PYG{p}{.}\PYG{n+nx}{name}\PYG{p}{)}\PYG{p}{)}\PYG{p}{\PYGZob{}}
    \PYG{n+nx}{layer}\PYG{p}{.}\PYG{n+nx}{name} \PYG{o}{=} \PYG{n+nx}{name} \PYG{o}{+} \PYG{l+s+s2}{\PYGZdq{}\PYGZus{}\PYGZdq{}} \PYG{o}{+} \PYG{n+nx}{i}\PYG{p}{;}
    \PYG{n+nx}{counter}\PYG{o}{++}\PYG{p}{;}
  \PYG{p}{\PYGZcb{}}
\PYG{p}{\PYGZcb{}}
\PYG{n+nx}{counter}\PYG{p}{;}
\end{sphinxVerbatim}

\item[{\sphinxcode{\sphinxupquote{createfile}}}] \leavevmode
\begin{sphinxVerbatim}[commandchars=\\\{\}]
\PYG{k+kd}{var} \PYG{n+nx}{text} \PYG{o}{=} \PYG{l+s+s2}{\PYGZdq{}\PYGZdq{}}\PYG{p}{;}
\PYG{k+kd}{var} \PYG{n+nx}{filePath} \PYG{o}{=} \PYG{n+nx}{Folder}\PYG{p}{.}\PYG{n+nx}{desktop}\PYG{p}{.}\PYG{n+nx}{fullName} \PYG{o}{+} \PYG{l+s+s2}{\PYGZdq{}/\PYGZus{}default.txt\PYGZdq{}}\PYG{p}{;}
\PYG{k+kd}{var} \PYG{n+nx}{file} \PYG{o}{=} \PYG{k}{new} \PYG{n+nx}{File}\PYG{p}{(}\PYG{n+nx}{filePath}\PYG{p}{)}\PYG{p}{;}
\PYG{c+c1}{//var file = File.saveDialog(\PYGZdq{}Choose a txt file\PYGZdq{},\PYGZdq{}*.txt*\PYGZdq{}, Folder.desktop);}
\PYG{k}{if}\PYG{p}{(}\PYG{n+nx}{file} \PYG{o}{===} \PYG{k+kc}{null}\PYG{p}{)}
  \PYG{n+nx}{file} \PYG{o}{=} \PYG{n+nx}{File}\PYG{p}{.}\PYG{n+nx}{saveDialog}\PYG{p}{(}\PYG{l+s+s2}{\PYGZdq{}Choose a txt file\PYGZdq{}}\PYG{p}{,}\PYG{l+s+s2}{\PYGZdq{}*.txt*\PYGZdq{}}\PYG{p}{,} \PYG{n+nx}{filePath}\PYG{p}{)}\PYG{p}{;}
\PYG{n+nx}{file}\PYG{p}{.}\PYG{n+nx}{open}\PYG{p}{(}\PYG{l+s+s2}{\PYGZdq{}w\PYGZdq{}}\PYG{p}{)}\PYG{p}{;}
\PYG{n+nx}{file}\PYG{p}{.}\PYG{n+nx}{writeln}\PYG{p}{(}\PYG{n+nx}{text}\PYG{p}{.}\PYG{n+nx}{toString}\PYG{p}{(}\PYG{p}{)}\PYG{p}{)}\PYG{p}{;}
\PYG{n+nx}{file}\PYG{p}{.}\PYG{n+nx}{close}\PYG{p}{(}\PYG{p}{)}\PYG{p}{;}
\end{sphinxVerbatim}

\end{description}


\subsection{Peacock Commands}
\label{\detokenize{modules/console/readme:peacock-commands}}
A list of all tab peacock commands code snippets.
\begin{description}
\item[{\sphinxcode{\sphinxupquote{marker}}}] \leavevmode
\begin{sphinxVerbatim}[commandchars=\\\{\}]
\PYG{n+nx}{markers}\PYG{p}{.}\PYG{n+nx}{addCompMarker}\PYG{p}{(}\PYG{n+nx}{comp}\PYG{p}{,} \PYG{k}{new} \PYG{n+nx}{Marker}\PYG{p}{(}\PYG{l+m+mi}{10}\PYG{p}{,} \PYG{p}{\PYGZob{}} \PYG{n+nx}{duration}\PYG{o}{:}\PYG{l+m+mf}{0.0} \PYG{p}{\PYGZcb{}}\PYG{p}{)}\PYG{p}{)}\PYG{p}{;}
\end{sphinxVerbatim}

\item[{\sphinxcode{\sphinxupquote{slice}}}] \leavevmode
\begin{sphinxVerbatim}[commandchars=\\\{\}]
\PYG{n+nx}{slices}\PYG{p}{.}\PYG{n+nx}{addCompSlice}\PYG{p}{(}\PYG{n+nx}{comp}\PYG{p}{,} \PYG{k}{new} \PYG{n+nx}{Slice}\PYG{p}{(}\PYG{l+m+mi}{5}\PYG{p}{,}\PYG{l+m+mi}{10}\PYG{p}{,} \PYG{p}{\PYGZob{}} \PYG{n+nx}{velocity}\PYG{o}{:}\PYG{l+m+mf}{1.0} \PYG{p}{\PYGZcb{}}\PYG{p}{)}\PYG{p}{)}\PYG{p}{;}
\end{sphinxVerbatim}

\end{description}


\section{Peacock Estk Tester}
\label{\detokenize{modules/estk_tester/readme:peacock-estk-tester}}\label{\detokenize{modules/estk_tester/readme::doc}}

\section{Peacock Expression Meister}
\label{\detokenize{modules/expressionmeister/readme:peacock-expression-meister}}\label{\detokenize{modules/expressionmeister/readme::doc}}
Before you can start using the BpmSlicer script you need to set the checkbox of the following setting to true.

\sphinxcode{\sphinxupquote{After Effects -\textgreater{} Preferences -\textgreater{} General -\textgreater{} Allow Scripts To Write Files And Access Network}}
(\sphinxcode{\sphinxupquote{Skripte können Dateien schreiben und haben Netwerkzugang}})

\begin{figure}[htbp]
\centering

\noindent\sphinxincludegraphics{{AE_Preferences1}.png}
\end{figure}


\section{Peacock Favorites}
\label{\detokenize{modules/favorites/readme:peacock-favorites}}\label{\detokenize{modules/favorites/readme::doc}}

\section{Peacock File Search}
\label{\detokenize{modules/filesearch/readme:peacock-file-search}}\label{\detokenize{modules/filesearch/readme::doc}}

\section{Peacock InDesign Utility}
\label{\detokenize{modules/in_utils/readme:peacock-indesign-utility}}\label{\detokenize{modules/in_utils/readme::doc}}

\section{Peacock Jsx Binary Converter}
\label{\detokenize{modules/jsxbinaryconverter/readme:peacock-jsx-binary-converter}}\label{\detokenize{modules/jsxbinaryconverter/readme::doc}}

\section{Peacock Keyframes}
\label{\detokenize{modules/keyframes/readme:peacock-keyframes}}\label{\detokenize{modules/keyframes/readme::doc}}

\section{Peacock Layertools}
\label{\detokenize{modules/layertools/readme:peacock-layertools}}\label{\detokenize{modules/layertools/readme::doc}}

\section{Peacock Library}
\label{\detokenize{modules/library/readme:peacock-library}}\label{\detokenize{modules/library/readme::doc}}

\section{Peacock Logger}
\label{\detokenize{modules/logger/readme:peacock-logger}}\label{\detokenize{modules/logger/readme::doc}}

\section{Peacock Markers}
\label{\detokenize{modules/markers/readme:peacock-markers}}\label{\detokenize{modules/markers/readme::doc}}

\section{Peacock Midiparser}
\label{\detokenize{modules/midiparser/readme:peacock-midiparser}}\label{\detokenize{modules/midiparser/readme::doc}}

\section{Peacock Performance}
\label{\detokenize{modules/performance/readme:peacock-performance}}\label{\detokenize{modules/performance/readme::doc}}

\section{Peacock Preferences}
\label{\detokenize{modules/preferences/readme:peacock-preferences}}\label{\detokenize{modules/preferences/readme::doc}}

\section{Peacock Progressbar}
\label{\detokenize{modules/progressbar/readme:peacock-progressbar}}\label{\detokenize{modules/progressbar/readme::doc}}

\section{Peacock Photoshop Utility}
\label{\detokenize{modules/ps_utils/readme:peacock-photoshop-utility}}\label{\detokenize{modules/ps_utils/readme::doc}}

\section{Peacock Session}
\label{\detokenize{modules/session/readme:peacock-session}}\label{\detokenize{modules/session/readme::doc}}

\section{Peacock Slices}
\label{\detokenize{modules/slices/readme:peacock-slices}}\label{\detokenize{modules/slices/readme::doc}}

\section{Peacock TextParser}
\label{\detokenize{modules/textparser/readme:peacock-textparser}}\label{\detokenize{modules/textparser/readme::doc}}

\section{Peacock Time Analyser}
\label{\detokenize{modules/timeanalyser/readme:peacock-time-analyser}}\label{\detokenize{modules/timeanalyser/readme::doc}}

\section{Peacock Transitions}
\label{\detokenize{modules/transitions/readme:peacock-transitions}}\label{\detokenize{modules/transitions/readme::doc}}

\section{Peacock UI Utility}
\label{\detokenize{modules/ui_utils/readme:peacock-ui-utility}}\label{\detokenize{modules/ui_utils/readme::doc}}

\section{Peacock Univarsal Utility}
\label{\detokenize{modules/utils/readme:peacock-univarsal-utility}}\label{\detokenize{modules/utils/readme::doc}}

\section{Troubleshooting}
\label{\detokenize{troubleshooting:troubleshooting}}\label{\detokenize{troubleshooting::doc}}
\begin{sphinxShadowBox}
\sphinxstyletopictitle{Table of Contents}
\begin{itemize}
\item {} 
\phantomsection\label{\detokenize{troubleshooting:id1}}{\hyperref[\detokenize{troubleshooting:troubleshooting}]{\sphinxcrossref{Troubleshooting}}}
\begin{itemize}
\item {} 
\phantomsection\label{\detokenize{troubleshooting:id2}}{\hyperref[\detokenize{troubleshooting:my-working-environment}]{\sphinxcrossref{My working environment:}}}
\begin{itemize}
\item {} 
\phantomsection\label{\detokenize{troubleshooting:id3}}{\hyperref[\detokenize{troubleshooting:errors}]{\sphinxcrossref{Errors}}}

\end{itemize}

\end{itemize}

\end{itemize}
\end{sphinxShadowBox}


\subsection{My working environment:}
\label{\detokenize{troubleshooting:my-working-environment}}\begin{itemize}
\item {} 
macOS High Sierra Version 10.13.3

\item {} 
Modellname: MacBook Pro

\item {} 
After Effects CC

\item {} 
Version: 2017.0

\item {} 
14.0.0.207

\end{itemize}


\subsubsection{Errors}
\label{\detokenize{troubleshooting:errors}}
\begin{sphinxadmonition}{error}{Error:}
\sphinxstylestrong{Error:}
ERROR: After Effects Warnung Rückgängig machen nicht übereinstimmender Gruppen: es wird versucht, den Fehler zu beheben.

\sphinxstylestrong{Description:}
I create composition markers by hand, read them into markersArray, add markersArray to another layer, if I then move the layer the error happens and all markers of the moved layer will get removed.
\end{sphinxadmonition}

\begin{sphinxadmonition}{error}{Error:}
\sphinxstylestrong{Error:}
Zuletzt protokollierte Meldung: \textless{}140736042881856\textgreater{} \textless{}BEE\_WorkQueue\textgreater{} \textless{}5\textgreater{} BEE\_Project::TimestampGetNext ZANZIBAR-3: cannot produce timestamp, frozen=0, open=0. Absturzprotokoll wird erstellt. Dies kann einige Minuten dauern.

\sphinxstylestrong{Description:}
I created a slice with ‘slices.createCompSlice(comp, new Slice(5,10));’ and moved the marker by dragging it to the left.
\end{sphinxadmonition}

\begin{sphinxadmonition}{error}{Error:}
\sphinxstylestrong{Error:}
If the project is saved either with autosave or with cmd+s the script is crashing and all the custom gui elements are disappearing.

\sphinxstylestrong{Description:}
Actually the next day after restarting the computer and after effects this error doesn’t happen in the beginning.
\end{sphinxadmonition}


\section{Work in Progress}
\label{\detokenize{work_in_progress:work-in-progress}}\label{\detokenize{work_in_progress::doc}}
\begin{sphinxShadowBox}
\sphinxstyletopictitle{Table of Contents}
\begin{itemize}
\item {} 
\phantomsection\label{\detokenize{work_in_progress:id1}}{\hyperref[\detokenize{work_in_progress:work-in-progress}]{\sphinxcrossref{Work in Progress}}}
\begin{itemize}
\item {} 
\phantomsection\label{\detokenize{work_in_progress:id2}}{\hyperref[\detokenize{work_in_progress:recent-searches}]{\sphinxcrossref{Recent Searches}}}

\item {} 
\phantomsection\label{\detokenize{work_in_progress:id3}}{\hyperref[\detokenize{work_in_progress:in-progress}]{\sphinxcrossref{In Progress}}}

\end{itemize}

\end{itemize}
\end{sphinxShadowBox}


\subsection{Recent Searches}
\label{\detokenize{work_in_progress:recent-searches}}\begin{itemize}
\item {} 
\sphinxhref{http://estk.aenhancers.com/4\%20-\%20User-Interface\%20Tools/control-objects.html\#progressbar}{Progressbar estk.aenhancers}

\item {} 
\sphinxhref{http://estk.aenhancers.com/4\%20-\%20User-Interface\%20Tools/control-objects.html\#shortcutkey}{ShortcutKey estk.aenhancers}

\item {} 
\sphinxhref{http://estk.aenhancers.com/4\%20-\%20User-Interface\%20Tools/event-handling.html}{Event Handling estk.aenhancers}

\item {} 
\sphinxhref{http://docs.aenhancers.com/layers/layer/\#layer-applypreset}{Layer applyPreset estk.aenhancers}

\item {} 
\sphinxhref{http://docs.aenhancers.com/layers/avlayer/\#avlayer-autoorient}{AVLayer autoOrient estk.aenhancers}

\item {} 
\sphinxhref{http://docs.aenhancers.com/layers/avlayer/\#avlayer-source}{AVLayer source estk.aenhancers}

\item {} 
\sphinxhref{http://docs.aenhancers.com/layers/avlayer/\#avlayer-replacesource}{AVLayer replaceSource estk.aenhancers}

\item {} 
\sphinxhref{http://docs.aenhancers.com/layers/avlayer/\#avlayer-sourcerectattime}{AVLayer sourceRectAtTime estk.aenhancers}

\item {} 
\sphinxhref{https://forums.adobe.com/thread/1240406}{NumericEditKeyboardHandler}

\item {} 
\sphinxhref{https://forums.adobe.com/thread/2591212}{Addeventlistener vs OnClick attribute}

\item {} 
\sphinxhref{https://www.adobeexchange.com/creativecloud.details.2450.script-console.html}{Script Console Script}

\end{itemize}


\subsection{In Progress}
\label{\detokenize{work_in_progress:in-progress}}


\renewcommand{\indexname}{Index}
\printindex
\end{document}